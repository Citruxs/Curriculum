\documentclass[]{kyvernitis-resume}
\fullname{Andrés Limón Cruz}
\jobtitle{Científico de datos Jr.}


\begin{document}
\resumeheader
{\email{andreslcruz26@gmail.com}}
{\phone{+52 56 3400 8301}}
{\linkedin{Andrés Limón Cruz}}




\begin{section}{Educacíón}
    \begin{subsectionnobullet}{Licenciatura}{Matemáticas Aplicadas}{Universidad Nacional Autónoma de México}{Sep 2021 -- Jun 2025}
    \item Facultad de Ciencias, Ciudad de México
    \end{subsectionnobullet}
\end{section}

\begin{section}{Experiencia}
	\begin{subsection}{DataLab Mx}{Cientifico de datos}{Dic 2024 - Actual}{Ciudad de México, México}
		\item Implementación de modelos de aprendizaje supervisado y no supervisado sobre conjuntos de datos de ventas comerciales para la toma de decisiones y creación de estrategias basadas en datos
		
		\item Implementación de modelos de series de tiempo estilo SARIMA y uso de herramientas especializadas (Prophet, Kats) para realizar forecasting de ventas anuales
		
	\end{subsection}
	
    \begin{subsection}{Laboratorio de modelación matemática}{Estudiante}{Ago 2022 - Dic 2022}{Ciudad de México, México}
        \item Implementación en Python, análisis y crítica de algoritmos (Mínimos cuadrados, Dijkstra, factorización LU y QR, Page Rank)
    \end{subsection}
    
    \begin{subsection}{Laboratorio de investigación de operaciones}{Estudiante}{Feb 2023 -- Jun 2023}{Ciudad de México, México}
        \item Implemetación en Python de algoritmos en investigación de operaciones (Esquina noroeste, Prim, Algoritmo PERT)
    \end{subsection}
	
	\begin{subsection}{Laboratorio de Probabilidad y Estadística}{Estudiante}{Ago 2023 -- Jun 2024}{Ciudad de México, México}
		\item Simulación de fenómenos probabilísticos a través de métodos computacionales para sus aplicaciones en Machine Learning e Inteligencia artificial (Simulación de variables aleatorias, Cadenas de Markov via Monte Carlo, Monte Carlo Crudo, Bootstrap, Movimiento Browniano, Procesos de Renovación)
		
		\item Realización de modelos lineales simples y múltiples con ayuda de R; Con aplicaciones en finanzas y medicina (Pruebas de hipótesis, ANOVA, ANCOVA)
		
		\item Realización de pruebas de hipótesis no paramétricas sobre poblaciones finitas con ayuda de R 
	\end{subsection}
    
    
\end{section}

\begin{section}{Proyectos}
\begin{subsectionnobullet}{Machine Learning}{Estudiante}{2023}{Coursera}
    \item Descripción: Aplicación de modelos de aprendizaje automático para el análisis de datos (Detección de cáncer, predicción de tendencias económicas, Predicción de clasificación de clientes, motores de recomendación).
    \item Herramientas usadas: Python, SQL, Visual Studio, Regresión, Clasfificación, Scipy, Numpy, Scikit Learn, Clustering.
\end{subsectionnobullet}

\begin{subsectionnobullet}{Análisis de datos}{Estudiante}{2023}{Coursera}
    \item Descripción: Aplicación de la metodología de análisis de datos de Google (Preguntar, Preparar, Procesar, Analizar, Compartir, Actuar) en conjuntos de datos empresariales para la toma de decisiones.
    \item Herramientas usadas: R, SQL, BigQuery, Rstudio, Tidyverse, Excel.
\end{subsectionnobullet}

\newpage

\begin{subsectionnobullet}{GAINS}{Estadístico}{2024}{SIAFI}
	\item Descripción: Extracción y limpieza de datos sobre portales de noticias, foros de internet y redes sociales para el entrenamiento sobre modelos largos de lenguaje (LLM y Transformers) que permitan conocer las tendencias del mercado sobre distintas acciones y criptomonedas, permitiendo al usuario una toma de decisiones asistida por inteligencia artificial.
	\item Herramientas usadas: Python, SQL, Visual Studio, Clasfificación, Numpy, Scikit Learn, Tranformers, Beautiful Soup, Java Script, Docker
\end{subsectionnobullet}

\end{section}

\begin{section}{Certificaciones}
    \begin{subsectionnobullet}{IBM}{IBM Data Science}{Jul 2023 - Sin fecha de expiración}{}
        \item Limpieza, análisis, visualización de datos y herramientas clave para machine learning e inteligencia artificial (Scikit-Learn, Scipy, Numpy, Pandas, Folium, Seaborn).
    \end{subsectionnobullet}
    
    \begin{subsectionnobullet}{Google}{Análisis de datos de Google}{Jun 2023 - Sin fecha de expiración}{}
        \item Destrezas analíticas (limpieza, análisis y visualización de datos) y herramientas clave (hojas de cálculo, SQL, programación en R, Tableau).
    \end{subsectionnobullet}
\end{section}

\sectiontable{Habilidades técnicas}{
    \entry{Lenguajes de programación}{Python (3 años), R (3 años), SQL (2 años), Excel (3 años), LateX (3 años), Java (2 años), Julia (1 año), C++ (1 año)}
}

\sectiontable{Otras habilidades}{
    \entry{Lenguajes}{Español (Nativo), Inglés (B1)}
}

\end{document}
